\documentclass[a4paper]{article}

\usepackage[utf8]{inputenc}
\usepackage{lmodern}
\usepackage{microtype}
\usepackage{amsfonts,amsmath,amssymb,amsthm,booktabs,color,enumitem,graphicx,listings,float}
\usepackage[pdftex,hidelinks]{hyperref}
\usepackage[finnish]{babel}
\selectlanguage{finnish}
\usepackage[fixlanguage]{babelbib}

\title{Pro gradu--suunnitelma: \\ Eheyden säilyminen tiedostojärjestelmän kaatumistilanteissa}
\author{Tuomas Tynkkynen}

\begin{document}
\maketitle

Tiedostojärjestelmät ovat käyttäjille ja ohjelmistoille tapa tallettaa tietoa pysyvään muistiin, kuten kiintolevylle.
Käytännössä kaikissa käyttöjärjestelmissä tiedostojen käsittelyyn vaaditut rajapinnat sekä käyttöliittymä
sisältyvät itse käyttöjärjestelmään.
Esimerkiksi Unix-pohjaisissa käyttöjärjestelmissä tiedostojen hallinta tapahtuu POSIX-standardissa
määritellyillä systeemikutsuilla, esimerkiksi \texttt{open}, \texttt{read}, \texttt{write}, \texttt{stat}, \texttt{unlink}, \texttt{mkdir}
sekä lukuisia muita.

Tiedostojen levylle kirjoituksen aikana on mahdollista että kirjoitus keskeytyy jostakin odottamattomasta syystä.
Esimerkiksi ohjelmassa itsesään voi olla virhe ja se voi kaatua kesken tiedoston kirjoituksen.
Sähkökatkoksen yhteydessä tietokoneen virta voi katketa,
tai käyttäjä saattaa irroittaa siirrettävän median (kuten USB-tikun) tietokoneesta tahallisesti tai tahattomasti.
Tällaisissa poikkeustilanteissa on mahdollista että tietoa katoaa tai korruptoituu,
mikä ei luonnollisesti ole toivottavaa.
Tämän välttämiseksi on kuitenkin kehitelty erilaisia mekanismeja
joilla sekä sovellukset että tiedostojärjestelmät voivat säilyä eheänä kaatumisista huolimatta.
Aion gradutyössäni siis käsitellä näitä menetelmiä eheyden säilymiseen tiedostojärjestelmien kaatumistilanteissa.
Pääasiallinen tutkielman tekotapa on kirjallisuuskatsaus kolmesta eri osa-alueesta:
eheys sovellusohjelmoijan näkökulmasta,
tekniikat tiedostojärjestelmän sisäisen metadatan turvaamiseksi
sekä itse tiedostojärjestelmäajuritoteutusten testaaminen kaatumistilanteissa.
Muut virhetilanteet kuin kaatumiset,
esimerkiksi bittivirheet tai koko levyn hajoaminen RAID-järjestelmässä
jätetään ainakin alustavasti aihealueen ulkopuolelle mutta on mahdollinen laajentamisen aihe.

Yksi lähestymissuunta on eheyden säilyminen sovellusten kannalta.
Otetaan esimerkiksi graafinen tekstieditori.
Kun muokattua olemassaolevaa tiedostoa tallennetaan,
on toivottavaa että tallentaminen tapahtuu atomisesti,
eli virhetilanteesta huolimatta levylle täytyy päätyä joko muokattu versio kokonaisena,
tai tiedoston tulee säilyä muuttumattomana.
Tällaista operaatiota ei löydy suoraan POSIX-standardista,
vaan sovelluskehittäjän täytyy itse toteuttaa tiedoston tallennuksen atomisuus useasta POSIX-tiedosto-operaatiosta koostamalla.
Toimivan ratkaisun löytäminen ei ole välttämättä itsestäänselvää,
sillä paremman suorituskyvyn saavuttamiseksi suurin osa POSIX-funktiokutsusta ei takaa operaation olevan persistentisti kirjoitettuna levylle synkronisesti.
Esimerkiksi tiedostoon \texttt{write()}-kutsulla kirjoitettu data päätyy modernissa käyttöjärjestelmässä ensin välimuistiin.
Samaten monet hakemistoja muokkaavat kutsut eivät ole oletuksena pysyviä (durable)
eivätkä välttämättä päädy levylle samassa järjestyksessä kuin missä sovellus teki ne.
Tätä aihetta käsittelee artikkeli ``All File Systems Are Not Created Equal: On the Complexity of Crafting Crash-consistent Applications''~\cite{PosixDataConsistency},
jossa empiirisesti testataan joitakin Linux-sovelluksia erityisellä työkalulla.
Usean sovelluksen todetaan korruptoivan dataa yleisillä Linux-tiedostojärjestelmillä,
mutta myös että useat käytännössä oikein toimivat sovellukset tekevät oletuksia tiedostojärjestemäoperaatioiden eheydestä joita POSIX ei lupaa.
Tietyt tiedostojärjestelmät siis tarjoavat (ei välttämättä tarkoituksenmukaisesti) vahvempia lupauksia eheydestä,
mikä voi aiheuttaa sovelluksen korruptoivan dataa tietyillä tiedostojärjestelmillä mutta toimivan oikein joillain toisilla.
Esimerkiksi aikoinaan Linux-jakeluiden siirtymä \texttt{ext3}-tiedestojärjestelmästä sen seuraajaan \texttt{ext4}:ään tuotti ongelmia,
sillä lukuisat tosimaailman sovellukset toimivat oikein \texttt{ext3}:lla mutta ei enää \texttt{ext4}:llä.

Toinen tiedostojärjestelmien eheyden osa-alue on tiedostojärjestelmän sisäisen metadatan konsistenssiuden säilyminen.
Tiedostojärjestelmän sisäisessä toteutuksessa on tyypillisesti useita levyllä sijaitsevia tietorakenteita joilla ovat omat invarianttinsa.
Esimerkiksi hakemistorakenteen täytyy muodostaa yhtenäinen puu,
eli hakemistorakenteessa ei saa esiintyä syklejä ja jokaiseen tiedostoon pitää voida kulkea juurihakemistosta.
Yksittäisen hakemiston toteutuksena voidaan käyttää esimerkiksi B-puuta jossa avaimena käytetään tiedoston nimeä,
ja jolle luonnollisesti täytyy päteä B-puulle ominaiset järjestysinvariantit.
Monet POSIX-tiedostojärjestelmäoperaatiot sisältävät vaatimuksia mitä tiedostojärjestelmän pitää toteuttaa.
Esimerkiksi tiedoston siirtäminen tai uudelleennimeäminen \texttt{rename()}-kutsulla täytyy olla atominen,
ja tiedoston voi siirtää eri alihakemistoon kuin missä se alunperin oli.
Jos tällaisen siirto-operaation lähde- ja kohdehakemistot sijaitsevat eri levylohkoilla,
täytyisi kahta levylohkoa päivittää samaan aikaan mikä ei suoraan ole mahdollista tehdä lohkolaitteella.
Samaten esimerkiksi tiedoston poistaminen tyypillisesti vaatii useamman tietorakenteen päivitystä atomisesti,
kuten tiedoston käyttämän tilan vapauttaminen sekä tiedoston nimen poistamisen hakemistosta.
Tämänkaltaisten ongelmien ratkaisuun on neljä laaja-alaista lähestymistapaa:
kirjaavat tiedostojärjestelmät (journaling) kuten \texttt{ext3}~\cite{Ext2Journal},
copy on write-pohjaiset tiedostojärjestelmät kuten \texttt{btrfs}~\cite{Btrfs},
lokipohjaiset tiedostojärjestelmät kuten \texttt{Sprite LFS}~\cite{SpriteLfs} sekä "soft updates"~\cite{SoftUpdates}-pohjautuvat tiedostojärjestelmät kuten Berkeley FFS.
Näillä menetelmillä on kaikilla erilaisia hyviä ja huonoja puolia muutenkin kuin eheyden kannalta,
esimerkiksi lokipohjaisilla tiedostojärjestelmillä on mahdollista ottaa
pienellä vaivalla snapshoteja tiedostojärjestelmästä,
mikä helpottaa esimerkiksi inkrementaalisten varmuuskopioiden tekemistä.

Kolmas suuntaus on itse tiedostojärjestelmätoteutusten oikeellisuuden tutkiminen.
Erityisesti Linux-tiedostojärjestelmäajureista löytyy kohtuullinen määrä tutkimuksia lähdekoodin avoimuuden takia.
Yleinen tapa tutkia ajurin tai tiedostojärjestelmätyökalun tutkimista kaatumistilanteissa on asentaa lohkolaiteajurin ympärille komponentti
joka kirjaa lokitiedostoon kaikki kirjoitukset ylös lokitiedostoon.
Tästä lokitiedostosta voi jälkikäteen muodostaa kaikki tilat joissa lohkolaitteen on mahdollista olla kaatumisen jälkeen,
jolloin jokaiselle kohdatulle kaatumistilalle voidaan varmistaa esimerkiksi tiedostojärjestelmän sisäisten invarianttien säilyminen.
Tähän menetelmään on yhdistetty esimerkiksi mallintarkastusta (model checking) erinäisissä muodoissa,
muun muassa artikkelissa ``Using Model Checking to Find Serious File System Errors''~\cite{ModelChecking}.

\bibliographystyle{babalpha-lf}
\bibliography{lahteet}

\end{document}
