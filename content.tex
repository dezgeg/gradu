\section{Johdanto}

\section{POSIX-tiedostojärjestelmärajapinnat}

Aikoinaan oli käytössä useita Unix-pohjaisia käyttöjärjestelmiä jotka olivat periaatteessa samankaltaisia mutta käytännössä toimivat hieman eri tavalla.
Sovellusohjelmoijan kannalta tämä hankaloitti huomattavasti monella eri Unix-järjestelmällä toivivien ohjelmien toteutusta.
Tämän ratkaisemiseksi kehiteltiin POSIX-standardi~\ref{PosixSpec} yhtenäistämään eri Unix-toteutusten tarjoamat rajapinnat.
Ensimmäinen versio standardista julkaistiin vuonna 1988.
POSIX-standardi määrittelee muun muassa komentorivityökaluja sekä C-kielisiä rajapintoja muun muassa tiedostojen, säikeiden ja prosessien hallintaan.
Käytännössä mitään Linux-jakelua ei virallisesti ole sertifioitu POSIX-yhteensopivaksi,
mutta käytännössä kehittäjät tähtäävät POSIX-yhteensopivuuteen.

Luotettavien ja kaatumisen kestävien sovellusten luominen POSIXissa on suhteellisen monimutkaista.
Esitelläänkin seuraavaksi POSIXin tarjoamat C-rajapinnat tiedostojen käsittelyyn.

\subsection{Tiedostonimet}
POSIX-tiedostojärjestelmät tukevat hakemistoja, eli tiedostojärjestelmä muodostaa hierarkisen puurakenteen.
Tiedostopoluissa hakemistojen erottimena toimii kauttaviiva eli \texttt{/}-merkki.
\texttt{/}-merkillä alkavat tiedostopolut ovat \emph{absoluuttisia} polkuja,
muut polut ovat \emph{suhteellisia} polkuja.
Jokaisella ajonaikaisella prosessilla on oma nykyinen hakemistonsa (current directory),
jonka suhteen suhteelliset polut tulkitaan.
Esimerkiksi prosessilla jonka nykyinen hakemisto on \texttt{/home/tuomas},
viittaisi polku \texttt{Downloads/kuva.jpg} tiedostoon \texttt{/home/tuomas/Downloads/kuva.jpg}.
POSIXissa on kaksi erityistä tiedostonimeä: \texttt{.} ja \texttt{..} joiden voidaan mieltää löytyvän jokaisesta hakemistosta.
Näistä \texttt{.} viittaa aina siihen hakemistoon jossa se sijaitsee,
esimerkiksi polku \texttt{/home/tuomas/.} viittaa samaan hakemistoon kuin polku \texttt{/home/tuomas}.
Niinikään \texttt{.}-nimeä voi käyttää missä tahansa kohtaa polkua,
eli esimerkiksi \texttt{/home/./tuomas} tai \texttt{/home/././tuomas/././.} viittaavat edelleen samaan hakemistoon.
Nimi \texttt{..} taas viittaa ylempään tasoon hakemistorakenteessa.

% http://pubs.opengroup.org/onlinepubs/9699919799/functions/open.html#tag_16_357
\begin{verbatim}
int open(const char *path, int oflag, ...);
\end{verbatim}

% http://pubs.opengroup.org/onlinepubs/9699919799/functions/close.html#tag_16_67
\begin{verbatim}
int close(int fildes);
\end{verbatim}

% http://pubs.opengroup.org/onlinepubs/9699919799/functions/read.html#tag_16_474
\begin{verbatim}
ssize_t read(int fildes, void *buf, size_t nbyte);
\end{verbatim}

% http://pubs.opengroup.org/onlinepubs/9699919799/functions/write.html#tag_16_685
\begin{verbatim}
ssize_t write(int fildes, const void *buf, size_t nbyte);
\end{verbatim}

% http://pubs.opengroup.org/onlinepubs/9699919799/functions/fstat.html#tag_16_173
\begin{verbatim}
int fstat(int fildes, struct stat *buf);
\end{verbatim}

% dev_t st_dev            Device ID of device containing file
% ino_t st_ino            File serial number.
% mode_t st_mode          Mode of file (see below).
% nlink_t st_nlink        Number of hard links to the file.
% uid_t st_uid            User ID of file.
% gid_t st_gid            Group ID of file.
% dev_t st_rdev           Device ID (if file is character or block special).
% off_t st_size           For regular files, the file size in bytes.
%                         For symbolic links, the length in bytes of the
%                         pathname contained in the symbolic link.
% struct timespec st_atim Last data access timestamp.
% struct timespec st_mtim Last data modification timestamp.
% struct timespec st_ctim Last file status change timestamp.
% blksize_t st_blksize    A file system-specific preferred I/O block size
%                         for this object. In some file system types, this
%                         may vary from file to file.
% blkcnt_t st_blocks      Number of blocks allocated for this object.
%
% http://pubs.opengroup.org/onlinepubs/9699919799/functions/chmod.html#tag_16_58
\begin{verbatim}
int chmod(const char *path, mode_t mode);
\end{verbatim}

% http://pubs.opengroup.org/onlinepubs/9699919799/functions/chown.html#tag_16_59
\begin{verbatim}
int chown(const char *path, uid_t owner, gid_t group);
\end{verbatim}

% http://pubs.opengroup.org/onlinepubs/9699919799/functions/link.html#tag_16_293
\begin{verbatim}
int link(const char *path1, const char *path2);
\end{verbatim}

% http://pubs.opengroup.org/onlinepubs/9699919799/functions/unlink.html#tag_16_635
\begin{verbatim}
int unlink(const char *path);
\end{verbatim}

% http://pubs.opengroup.org/onlinepubs/9699919799/functions/rename.html#tag_16_487
\begin{verbatim}
int rename(const char *old, const char *new);
\end{verbatim}

% http://pubs.opengroup.org/onlinepubs/9699919799/functions/readdir.html#tag_16_475
\begin{verbatim}
struct dirent *readdir(DIR *dirp);
\end{verbatim}

% http://pubs.opengroup.org/onlinepubs/9699919799/functions/readlink.html#tag_16_476
\begin{verbatim}
ssize_t readlink(const char *restrict path, char *restrict buf, size_t bufsize);
\end{verbatim}

% http://pubs.opengroup.org/onlinepubs/9699919799/functions/ftruncate.html
\begin{verbatim}
int ftruncate(int fildes, off_t length);
\end{verbatim}

% http://pubs.opengroup.org/onlinepubs/9699919799/functions/fdopendir.html#tag_16_127
\begin{verbatim}
DIR *opendir(const char *dirname);
\end{verbatim}

% http://pubs.opengroup.org/onlinepubs/9699919799/functions/closedir.html
\begin{verbatim}
int closedir(DIR *dirp);
\end{verbatim}

% http://pubs.opengroup.org/onlinepubs/9699919799/functions/lseek.html#tag_16_310
\begin{verbatim}
off_t lseek(int fildes, off_t offset, int whence);
\end{verbatim}

\section{Yhteenveto}
