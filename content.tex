\section{Johdanto}

\section{POSIX-tiedostojärjestelmärajapinnat}

Aikoinaan oli käytössä useita Unix-pohjaisia käyttöjärjestelmiä jotka olivat periaatteessa samankaltaisia mutta käytännössä toimivat hieman eri tavalla.
Sovellusohjelmoijan kannalta tämä hankaloitti huomattavasti monella eri Unix-järjestelmällä toivivien ohjelmien toteutusta.
Tämän ratkaisemiseksi kehiteltiin POSIX-standardi~\ref{PosixSpec} yhtenäistämään eri Unix-toteutusten tarjoamat rajapinnat.
Ensimmäinen versio standardista julkaistiin vuonna 1988.
POSIX-standardi määrittelee muun muassa komentorivityökaluja sekä C-kielisiä rajapintoja muun muassa tiedostojen, säikeiden ja prosessien hallintaan.
Käytännössä mitään Linux-jakelua ei virallisesti ole sertifioitu POSIX-yhteensopivaksi,
mutta käytännössä kehittäjät tähtäävät POSIX-yhteensopivuuteen.

Luotettavien ja kaatumisen kestävien sovellusten luominen POSIXissa on suhteellisen monimutkaista.
Esitelläänkin seuraavaksi POSIXin tarjoamat C-rajapinnat tiedostojen käsittelyyn.

\section{Yhteenveto}
